\documentclass{article}
\usepackage{listings}

\title{Arch Installation Summary}
\author{Lucas de Macedo}

\begin{document}
\maketitle
\section{Disclaimer}
This is a personal guide to an arch install, made by me, which is being
constantly update and fixed as I learn new and better ways of setting up the
system, so do not blame me if something goes wrong with your installation.

Also, this guide is suited to my system and preferences, like language settings,
locale, preferred disk partition, windows manager, programs used, etc

By the way, there are 2 scripts in this repo that do all that is written here.
The \textbf{install.sh} one is used to set up and install arch, the the \textbf{setup.sh} is used
to install all the programs and configurations after the reboot.

\section{Before chroot}

\subsection{Verify Boot Mode}

\subsection{Keyboard Layout}
\begin{lstlisting}
  ls /usr/share/kbd/keymaps/**/*.map.gz
  loadkeys br-abnt2
\end{lstlisting}

\subsection{Network Configuration}
\begin{lstlisting}
  wifi-menu
  ping google.com
  ip address show
  systemctl stop dhcpdc@interface
\end{lstlisting}

\subsection{Date and Clock}
\begin{lstlisting}
  timedatectl set-ntp true
\end{lstlisting}

\subsection{Disk Partition}
\begin{enumerate}
  \item efi - 512 MB
  \item root - 20 or 30 GB
  \item swap - Half of RAM
  \item home - Rest of the space
\end{enumerate}

Create the partitions:

\begin{lstlisting}
  ldisk -l
  gdisk /dev/sdX

  EFI Partition:
  n (New Partition)
  1 (First Partition)
  Enter (Beginning of space)
  +512M (300 Megas)
  ef00 (Code for EFI System)

  Root Partition:
  n
  2
  Enter
  +30G (30 Gigas)
  Enter (Default Code, 8300)

  Swap Partition:
  n
  3
  Enter
  +8G
  8200 (Linux Swap)

  Home Partition:
  n
  4
  Enter
  Enter (Rest of the space)
  8302 (Linux /home)

  p (Check if everything is correct)
\end{lstlisting}

Format the partitions:

\begin{lstlisting}
  mkfs.fat -F32 /dev/sdX1
  mfks.ext4 /dev/sdX2
  mkswap /dev/sdX3}, \textbf{swapon /dev/sdX3
\end{lstlisting}

\subsection{Encrypt Filesystem}
TODO

\subsection{Filesystem Mounting}
\begin{lstlisting}
  mount /dev/sdX2 /mnt
  mkdir /mnt/home
  mount /dev/sdX4 /mnt/home
  mkdir /mnt/boot
  mkdir /mnt/boot/efi
  mount /dev/sdX1 /mnt/boot/efi
\end{lstlisting}

\subsection{Package Installation}
\begin{lstlisting}
  pacstrap -i /mnt base base-devel
  genfstab -U -p /mnt >> /mnt/etc/fstab
  arch-chroot /mnt
\end{lstlisting}

\section{Into chroot}
Set the root password, and install bootloader and headers:
\begin{lstlisting}
  passwd
  pacman -S grub efibootmgr dosfstools os-prober mtools linux-headers vim
\end{lstlisting}

\subsection{Timezone, location and keeb layout}
\begin{lstlisting}
  ln -sf /usr/share/zoneinfo/America/Fortaleza /etc/localtime
  vim /etc/locale.gen # Uncomment pt_BR.UTF-8
  locale-gen
  hwclock --systoch
  echo "LANG=pt_BR.UTF-8" >> /etc/localge.conf
  echo "KEYMAP=" >> /etc/vconsole.conf
\end{lstlisting}

\subsection{Hostname}
\begin{lstlisting}
  echo "hyperion" >> /etc/hostname
\end{lstlisting}

Add the following to \textbf{/etc/hosts}:
\begin{lstlisting}
  127.0.0.1 localhost
  ::1       localhost
  127.0.1.1 hyperion.localdomain hyperion
\end{lstlisting}

\subsection{Grub}
\begin{lstlisting}
  grub-install --target=x86_64-efi --bootloader-id=grub_uefi --recheck
  cp /usr/share/locale/en\@quot/LC_MESSAGES/grub.mo /boot/grub/locale/en.mo
  grub-mkconfig -o /boot/grub/grub.cfg
\end{lstlisting}

\subsection{Network Configuration}
\begin{lstlisting}
  pacman -S NetworkManager
\end{lstlisting}

\subsection{User Configuration}
\begin{lstlisting}
  visudo # edit and uncomment the WHEEL line
  useradd -m USERNAME -G wheel
\end{lstlisting}

\subsection{Exiting chroot}
\begin{lstlisting}
  exit
  umount -a
  reboot
\end{lstlisting}

\section{After reboot}

\subsection{Network}
\begin{lstlisting}
  systemctl start NetworkManager.service
  systemctl enable NetworkManager.service
  nmcli device wifi connect network password password
\end{lstlisting}


\subsection{Display and Graphic}
\begin{lstlisting}
  pacman -Ss xf86-video # check all the video drivers
  sudo pacman -S xf86-video-intel xf86-input-mouse xf86-input-keyboard
  sudo pacman -S xorg-server xorg-xinit xterm
  Xorg -configure
\end{lstlisting}

\subsection{Windows Manager}
\begin{lstlisting}
  sudo pacman -S i3 dmenu
  echo "exec i3" >> $HOME/.xinitrc
\end{lstlisting}

\subsection{Login Manager}
\begin{lstlisting}
  sudo pacman -S lightdm lightdm-gtk-greeter xorg-xerver-xephyr accountsservice
  systemctl enable lightdm.service
  systemctl start lightdm.service
\end{lstlisting}

\subsection{Sound}
\begin{lstlisting}
\end{lstlisting}

\subsection{Rest of the instalattion}
I use my script on my dotfiles to set up the res of the system to my liking

\end{document}
